%% Согласно ГОСТ Р 7.0.11-2011:
%% 5.3.3 В заключении диссертации излагают итоги выполненного исследования, рекомендации, перспективы дальнейшей разработки темы.
%% 9.2.3 В заключении автореферата диссертации излагают итоги данного исследования, рекомендации и перспективы дальнейшей разработки темы.
\textcolor{blue}{В ходе исследования была систематически разработана, теоретически обоснована и экспериментально верифицирована методология комплексной оценки неопределённости в задаче открытого распознавания (Open-Set Recognition, OSR), учитывающая два фундаментальных источника риска: качество входного сигнала и структуру галереи известных объектов.}  
\textcolor{gray}{Эта задача является критически важной для практического применения биометрических и других идентификационных систем, где недостаточная оценка достоверности принимаемых решений может привести к ложным срабатываниям с серьёзными последствиями.}  

\textcolor{blue}{В рамках работы были решены все поставленные задачи:}  
\textcolor{gray}{1. Проведён всесторонний анализ существующих подходов к оценке неопределённости в OSR, выявивших их ключевые ограничения — игнорирование галерею-осознанной неопределённости и уязвимость к тренировочному смещению.}  
\textcolor{gray}{2. Разработана математическая модель, основанная на байесовской теории, которая формально объединяет информацию о качестве входного сигнала и относительном положении эмбеддинга в пространстве галереи.}  
\textcolor{gray}{3. Предложен и реализован метод HolUE (Holistic Uncertainty Estimation), а также его компонент GalUE (Gallery-Aware Uncertainty Estimation), позволяющие детектировать все три типа ошибок OSR: ложное принятие (false acceptance), ложный отказ (false rejection) и ошибочную идентификацию (misidentification).}  
\textcolor{blue}{4. Проведена масштабная экспериментальная оценка на четырёх наборах данных (IJB-C, IJB-B, VoxBlink, Whale), которая подтвердила превосходство HolUE над современными методами, включая SCF, PFE, ScaleFace и AccScr, по метрике Prediction Rejection Ratio (PRR).}  
\textcolor{red}{5. Показано, что качество оценки неопределённости напрямую зависит от соответствия распределения обучающих и тестовых данных, и предложен метод снижения тренировочного смещения с помощью целенаправленных аугментаций.}  

\textcolor{gray}{Таким образом, все четыре положения, выносимые на защиту, полностью подтверждены как теоретически, так и экспериментально.}  
\textcolor{blue}{Научная новизна работы заключается в первом в литературе объединении двух независимых источников неопределённости в единой байесовской модели и введении понятия галерею-осознанной неопределённости как отдельной составляющей риска в системах OSR.}  
\textcolor{gray}{Практическая значимость подтверждена успешным применением метода HolUE в задачах распознавания лиц, идентификации дикторов и идентификации морских млекопитающих, что демонстрирует его доменную универсальность.}  

\textcolor{blue}{Достоверность результатов обеспечивается строгостью выводов, воспроизводимостью экспериментов, использованием общепринятых протоколов и сравнением с актуальными бейзлайнами.}  
\textcolor{gray}{Работа соответствует требованиям, предъявляемым к диссертациям на соискание учёной степени кандидата технических наук по специальности 1.2.1 — «Искусственный интеллект и машинное обучение».}  
\textcolor{red}{Полученные результаты являются законченным и завершённым этапом исследования, при этом открывают перспективы для дальнейшего развития в направлении мультимодальных систем, активного обучения и интеграции с большими языковыми моделями.}  

\textcolor{gray}{В заключение автор выражает глубокую благодарность научному руководителю Зайцеву Алексею Алексеевичу за поддержку, ценные замечания и постоянное внимание к работе.}  
\textcolor{blue}{Автор также благодарит коллег по Сколтеху и МФТИ за плодотворные обсуждения и помощь в подготовке материалов.}
