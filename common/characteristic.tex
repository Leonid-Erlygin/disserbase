Область искусственного интеллекта и машинного обучения активно развивается, находя всё более широкое применение в реальных системах, требующих надёжного принятия решений.
Одной из ключевых задач является \textit{открытое распознавание} (Open-Set Recognition, OSR), где система должна не только идентифицировать известные объекты, но и надёжно отвергать неизвестные.
Эта задача критически важна для обеспечения безопасности и надёжности систем биометрической идентификации, аудио- и видеоаналитики.
Однако, несмотря на значительные успехи в области глубокого обучения, современные системы часто неспособны корректно оценить свою уверенность в принятом решении, что приводит к серьёзным ошибкам: ложному принятию неизвестных объектов (false acceptance), ложному отказу в идентификации известных объектов (false rejection) или неправильной идентификации (misidentification).

Основной причиной этих ошибок является то, что большинство существующих методов оценки неопределённости сосредоточены исключительно на качестве входного сигнала (например, размытие, освещение, шум в аудио), игнорируя другую, не менее важную, составляющую — неопределённость, обусловленную структурой набора известных объектов (галереи).
Например, если сигнал на входе системы близок к нескольким классам в галерее или находится на границе между «принять» и «отклонить», система должна демонстрировать высокую степень неопределённости, даже если сам сигнал имеет хорошее качество.
Существующие методы, такие как Spherical Confidence Face (SCF) или Probabilistic Face Embeddings (PFE), не учитывают эту галерею-осознанную неопределённость, что делает их неполноценными для задач OSR.
Кроме того, эти методы часто обучаются на фильтрованных наборах данных, что приводит к тренировочному смещению и снижению их эффективности на реальных, разнообразных данных.

Целью данной работы является разработка и обоснование метода комплексной оценки неопределённости в задаче открытого распознавания, способного учитывать как неопределённость, связанную с качеством входного сигнала, так и неопределённость, обусловленную относительным положением эмбеддинга сигнала в пространстве галереи известных объектов.

Для достижения поставленной цели были решены следующие задачи:
\begin{enumerate}
    \item Исследовать и проанализировать существующие подходы к оценке неопределённости в задаче OSR, выявить их ключевые недостатки, связанные с игнорированием галерею-осознанной неопределённости и тренировочным смещением.
    \item Разработать математическую модель, объединяющую два источника неопределённости (качество сигнала и структура галереи) в единую меру, основанную на байесовской теории.
    \item Предложить и реализовать метод HolUE (Holistic Uncertainty Estimation), который на основе этой модели позволяет детектировать все три типа ошибок OSR: false acceptance, false rejection и misidentification.
    \item Провести всестороннее экспериментальное исследование предложенного метода на нескольких наборах данных (IJB-C, IJB-B, VoxBlink, Whale) и сравнить его с существующими методами оценки неопределённости.
    \item Исследовать влияние качества обучающих данных на точность оценки неопределённости и предложить методы снижения тренировочного смещения, в частности, путём использования аугментаций во время обучения.
\end{enumerate}
Научная новизна работы заключается в следующем:
\begin{enumerate}
    \item Впервые предложен и обоснован метод \textit{HolUE}, который на основе байесовской модели объединяет две независимые составляющие неопределённости в единую меру, позволяя детектировать все три типа ошибок OSR.
    \item Впервые в рамках задачи OSR предложен и реализован метод \textit{GalUE} (Gallery-Aware Uncertainty Estimation), который позволяет оценивать неопределённость, обусловленную структурой галереи, путём моделирования выхода системы как категориального распределения вероятностей.
    \item Было выполнено оригинальное исследование, показавшее, что тренировочное смещение, возникающее при использовании отфильтрованных наборов данных (например, MS1MV2), негативно влияет на качество оценки неопределённости, и предложен метод его снижения за счёт использования аугментаций во время обучения.
    \item Проведено всестороннее экспериментальное исследование, подтвердившее превосходство предложенного метода HolUE над существующими подходами на четырёх различных наборах данных, включая новые данные по идентификации китов и дельфинов (Whale dataset).
\end{enumerate}

Практическая значимость работы заключается в возможности внедрения предложенного метода в реальные системы для повышения их надёжности и безопасности.
Метод позволяет системе автоматически отфильтровывать сомнительные запросы, предлагая пользователю повторить попытку, что значительно снижает риск ложных срабатываний.
Кроме того, предложенный подход является доменно-агностичным и может быть применён в других областях, таких как распознавание голоса или анализ текстовых данных.

Методология и методы исследования. В работе использовались методы теории вероятностей и математической статистики, теории информации, глубокого обучения, а также методы компьютерного моделирования.
Для реализации и тестирования предложенных методов были использованы стандартные библиотеки машинного обучения (PyTorch)


Основные положения, выносимые на защиту:
\begin{enumerate}
    \item Комплексная оценка неопределённости в задаче открытого распознавания требует учёта двух независимых источников: неопределённости, связанной с качеством входного сигнала, и неопределённости, обусловленной структурой галереи известных объектов.
    \item Предложенный метод HolUE, основанный на байесовской модели и расхождении Кулбака–Лейблера, позволяет объединить эти два источника неопределённости в единую меру, которая эффективно детектирует все три типа ошибок OSR.
    \item Тренировочное смещение, вызванное использованием отфильтрованных наборов данных, негативно влияет на точность оценки неопределённости, и его можно снизить путём применения аугментаций во время обучения.
    \item Экспериментальные результаты на нескольких наборах данных (IJB-C, IJB-B, VoxBlink, Whale) показывают, что метод HolUE превосходит существующие методы оценки неопределённости, включая SCF, PFE и AccScr, по всем основным метрикам.
\end{enumerate}
Достоверность полученных результатов обеспечивается использованием строгих математических моделей, воспроизводимостью экспериментов на общедоступных наборах данных, а также сравнением с результатами, полученными другими авторами.
Все эксперименты проводились с использованием стандартных протоколов и метрик, принятых в сообществе.

Основные результаты работы докладывались на следующих конференциях и семинарах:
\begin{itemize}
    \item IEEE Access (проходит ревью);
\end{itemize}
Личный вклад автора заключается в постановке задачи, разработке математической модели, реализации алгоритмов, проведении экспериментов и анализе результатов.
Научный руководитель Зайцев А.А. участвовал в постановке задачи, обсуждении результатов и подготовке публикаций.

%\textcolor{gray}{Основные результаты по теме диссертации изложены в 3 печатных изданиях, 2 из которых изданы в журналах, рекомендованных ВАК, 1 — в периодическом научном журнале, индексируемом Web of Science и Scopus, 1 — в тезисах докладов.}