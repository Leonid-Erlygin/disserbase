{\actuality} Область искусственного интеллекта и машинного обучения активно развивается, находя все более широкое применение в реальных системах, требующих надежного принятия решений. Одной из ключевых задач является \textit{открытое распознавание} (Open-Set Recognition, OSR), где система должна не только идентифицировать известные объекты, но и надежно отвергать неизвестные. Эта задача критически важна для обеспечения безопасности и надежности систем биометрической идентификации, аудио- и видеоаналитики, а также в других приложениях, таких как идентификация животных по изображениям или звукам. Однако, несмотря на значительные успехи в области глубокого обучения, современные системы часто неспособны корректно оценить свою уверенность в принятом решении, что приводит к серьезным ошибкам: ложному принятию неизвестных объектов (false acceptance), ложному отказу в идентификации известных объектов (false rejection) или неправильной идентификации (misidentification).

Основной причиной этих ошибок является то, что большинство существующих методов оценки неопределенности сосредоточены исключительно на качестве входного сигнала (например, размытие, освещение, шум в аудио), игнорируя другую, не менее важную, составляющую — неопределенность, обусловленную структурой набора известных объектов (галереи). Например, если сигнал на входе системы близок к нескольким классам в галерее или находится на границе между «принять» и «отклонить», система должна демонстрировать высокую степень неопределенности, даже если сам сигнал имеет хорошее качество. Существующие методы, такие как Spherical Confidence Face (SCF) или Probabilistic Face Embeddings (PFE), не учитывают эту галерею-осознанную неопределенность, что делает их неполноценными для задач OSR. Кроме того, эти методы часто обучаются на фильтрованных наборах данных, что приводит к тренировочному смещению и снижению их эффективности на реальных, разнообразных данных.

\ifsynopsis
Этот абзац появляется только в~автореферате.
Для формирования блоков, которые будут обрабатываться только в~автореферате,
заведена проверка условия \verb!\!\verb!ifsynopsis!.
Значение условия задаётся в~основном файле документа (\verb!synopsis.tex! для
автореферата).
\else
Этот абзац появляется только в~диссертации.
Через проверку условия \verb!\!\verb!ifsynopsis!, задаваемого в~основном файле
документа (\verb!dissertation.tex! для диссертации), можно сделать новую
команду, обеспечивающую появление цитаты в~диссертации, но~не~в~автореферате.
\fi

{\aim} Целью данной работы является разработка и обоснование метода комплексной оценки неопределенности в задаче открытого распознавания, способного учитывать как неопределенность, связанную с качеством входного сигнала, так и неопределенность, обусловленную относительным положением эмбеддинга сигнала в пространстве галереи известных объектов.

{\tasks}
\begin{enumerate}
  \item Исследовать и проанализировать существующие подходы к оценке неопределенности в задаче OSR, выявить их ключевые недостатки, связанные с игнорированием галереи-осознанной неопределенности и тренировочным смещением.
  \item Разработать математическую модель, объединяющую два источника неопределенности (качество сигнала и структура галереи) в единую меру, основанную на байесовской теории.
  \item Предложить и реализовать метод HolUE (Holistic Uncertainty Estimation), который на основе этой модели позволяет детектировать все три типа ошибок OSR: false acceptance, false rejection и misidentification.
  \item Провести всестороннее экспериментальное исследование предложенного метода на нескольких наборах данных (IJB-C, IJB-B, VoxBlink, Whale) и сравнить его с существующими методами оценки неопределенности.
  \item Исследовать влияние качества обучающих данных на точность оценки неопределенности и предложить методы снижения тренировочного смещения, в частности, путем использования аугментаций во время обучения.
\end{enumerate}

{\novelty}
\begin{enumerate}
  \item Впервые предложен и обоснован метод \textit{HolUE}, который на основе байесовской модели объединяет две независимые составляющие неопределенности в единую меру, позволяя детектировать все три типа ошибок OSR.
  \item Впервые в рамках задачи OSR предложен и реализован метод \textit{GalUE} (Gallery-Aware Uncertainty Estimation), который позволяет оценивать неопределенность, обусловленную структурой галереи, путем моделирования выхода системы как категориального распределения вероятностей.
  \item Было выполнено оригинальное исследование, показавшее, что тренировочное смещение, возникающее при использовании отфильтрованных наборов данных (например, MS1MV2), негативно влияет на качество оценки неопределенности, и предложен метод его снижения за счет использования аугментаций во время обучения.
  \item Проведено всестороннее экспериментальное исследование, подтвердившее превосходство предложенного метода HolUE над существующими подходами на четырех различных наборах данных, включая новые данные по идентификации китов и дельфинов (Whale dataset).
\end{enumerate}

{\influence} Практическая значимость работы заключается в возможности внедрения предложенного метода в реальные системы для повышения их надежности и безопасности. Метод позволяет системе автоматически отфильтровывать сомнительные запросы, предлагая пользователю повторить попытку, что значительно снижает риск ложных срабатываний. Кроме того, предложенный подход является доменно-агностичным и может быть применен в других областях, таких как распознавание голоса, идентификация животных или анализ текстовых данных.

{\methods} Методология и методы исследования. В работе использовались методы теории вероятностей и математической статистики, теории информации, глубокого обучения, а также методы компьютерного моделирования. Для реализации и тестирования предложенных методов были использованы стандартные библиотеки машинного обучения (PyTorch, scikit-learn) и специализированные библиотеки для работы с биометрическими данными.

{\defpositions}
\begin{enumerate}
  \item Комплексная оценка неопределенности в задаче открытого распознавания требует учета двух независимых источников: неопределенности, связанной с качеством входного сигнала, и неопределенности, обусловленной структурой галереи известных объектов.
  \item Предложенный метод HolUE, основанный на байесовской модели и расхождении Кулбака-Лейблера, позволяет объединить эти два источника неопределенности в единую меру, которая эффективно детектирует все три типа ошибок OSR.
  \item Тренировочное смещение, вызванное использованием отфильтрованных наборов данных, негативно влияет на точность оценки неопределенности, и его можно снизить путем применения аугментаций во время обучения.
  \item Экспериментальные результаты на нескольких наборах данных (IJB-C, IJB-B, VoxBlink, Whale) показывают, что метод HolUE превосходит существующие методы оценки неопределенности, включая SCF, PFE и AccScr, по всем основным метрикам.
\end{enumerate}

% \todo{В папке Documents можно ознакомиться в решением совета из Томского ГУ
% в файле \verb+Def_positions.pdf+, где обоснованно даются рекомендации
% по формулировкам защищаемых положений.}

{\reliability} Достоверность полученных результатов обеспечивается использованием строгих математических моделей, воспроизводимостью экспериментов на общедоступных наборах данных, а также сравнением с результатами, полученными другими авторами. Все эксперименты проводились с использованием стандартных протоколов и метрик, принятых в сообществе.

{\probation}
Основные результаты работы докладывались на следующих конференциях и семинарах:
\begin{itemize}
    \item IEEE Access (проходит ревью)
    \item WACV 2025 (подана в качестве конфиденциальной заявки)
    \item Внутренние семинары в Сколтехе
\end{itemize}

{\contribution} Личный вклад. Автор принимал активное участие в постановке задачи, разработке математической модели, реализации алгоритмов, проведении экспериментов и анализе результатов. Научный руководитель Зайцев А.А. участвовал в постановке задачи, обсуждении результатов и подготовке публикаций.

{\publications} Основные результаты по теме диссертации изложены в 3 печатных изданиях, 2 из которых изданы в журналах, рекомендованных ВАК, 1 — в периодическом научном журнале, индексируемом Web of Science и Scopus, 1 — в тезисах докладов.