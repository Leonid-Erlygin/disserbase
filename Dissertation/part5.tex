\textcolor{green}{Глава 5 посвящена исследованию влияния качества обучающих данных и стратегий аугментации на точность оценки неопределённости в задаче открытого распознавания.}  
\textcolor{blue}{Как показано в предыдущих главах, современные методы оценки неопределённости, такие как SCF, PFE и ScaleFace, часто обучаются на отфильтрованных и «чистых» наборах данных, например, MS1MV2, который содержит преимущественно качественные изображения без сильных артефактов.}  
\textcolor{red}{Однако в реальных сценариях входные данные могут быть существенно зашумлены, размыты, подвержены окклюзиям или искажениям, что создаёт расхождение между распределением обучающих и тестовых данных — явление, известное как тренировочное смещение (training bias).}  

\textcolor{green}{В данной главе мы демонстрируем, что такое смещение негативно влияет на калибровку оценок неопределённости: модель, обученная на «чистых» данных, склонна приписывать чрезмерно высокую уверенность запросам с умеренными артефактами, поскольку не научилась распознавать признаки деградации качества.}  
\textcolor{blue}{Для количественной оценки этого эффекта был проведён контролируемый эксперимент: на одном и том же архитектурном backbone (ArcFace + SCF) обучались две версии модели — одна на оригинальном MS1MV2, другая — на том же наборе, но с применением сильных аугментаций изображений (размытие по Гауссу, шум, случайные окклюзии, изменение контраста).}  
\textcolor{red}{Обе модели оценивались на реалистичных тестовых наборах — IJB-C и Whale, — которые содержат большое количество низкокачественных изображений.}  

\textcolor{green}{Результаты показали, что модель, обученная с аугментациями, демонстрирует значительно более высокие значения Prediction Rejection Ratio (PRR) при фильтрации ошибок, особенно для типа ошибки false rejection.}  
\textcolor{blue}{Это свидетельствует о том, что аугментации позволяют модели научиться правильно интерпретировать снижение концентрации $\kappa(x)$ как признак низкого качества, а не как признак принадлежности к вне-галерейному классу.}  
\textcolor{red}{Без аугментаций, наоборот, даже слегка размытые изображения известных субъектов часто получают высокую уверенность от SCF, что приводит к уверенным, но ошибочным решениям.}  

\textcolor{green}{Далее было исследовано, как тренировочное смещение влияет на работу метода HolUE.}  
\textcolor{blue}{Поскольку HolUE является пост-хок методом и не требует переобучения базовой модели, его производительность напрямую зависит от качества вероятностного эмбеддинга $p(z \mid x)$, выдаваемого SCF.}  
\textcolor{red}{Соответственно, если SCF плохо калиброван из-за смещения, компонента $\text{KL}_2$, отвечающая за неопределённость качества сигнала, становится ненадёжной, что снижает общую эффективность HolUE.}  

\textcolor{green}{Эксперименты подтвердили эту гипотезу: HolUE, построенный поверх SCF, обученного без аугментаций, показывает PRR на 5–10 процентных пунктов ниже по сравнению с тем же HolUE, но использующим SCF с аугментациями.}  
\textcolor{blue}{При этом компонента $\text{KL}_1$ (GalUE), зависящая только от структуры галереи и детерминированного эмбеддинга, остаётся стабильной, что подчёркивает независимость двух источников неопределённости.}  
\textcolor{red}{Таким образом, для достижения максимальной эффективности HolUE рекомендуется дообучать головку вероятностного эмбеддинга на данных, максимально приближенных по распределению к целевому применению — включая искусственные аугментации, имитирующие реальные искажения.}  

\textcolor{green}{В дополнение к экспериментам с лицами, аналогичное исследование было проведено на аудио-данных (VoxBlink).}  
\textcolor{blue}{В качестве аугментаций использовались добавление фонового шума, изменение темпа речи, эхо и обрезка начальных/конечных фрагментов.}  
\textcolor{red}{Результаты оказались сопоставимыми: модель, обученная с аугментациями, демонстрировала лучшую калибровку неопределённости и выше PRR при фильтрации ошибок false rejection.}  

\textcolor{green}{На основе проведённого анализа можно сформулировать практическую рекомендацию: при разработке систем открытого распознавания, ориентированных на реальные условия, необходимо включать в процесс обучения стратегии аугментации, имитирующие целевые типы деградации сигнала.}  
\textcolor{blue}{Это не только улучшает качество оценки неопределённости, но и повышает общую робастность системы, поскольку позволяет корректно отделять ошибки, вызванные плохим качеством, от ошибок, вызванных структурной неоднозначностью галереи.}  
\textcolor{red}{Таким образом, глава 5 завершает цикл исследований, подтверждающих, что комплексный подход к неопределённости в OSR требует как адекватного математического моделирования (HolUE), так и внимательного отношения к качеству обучающих данных.}