\chapter{Обзор литературы}
\label{chap:related_work}

\section{Открытая установка распознавания}
\label{sec:osr}

Задача \textit{открытого распознавания} (Open-Set Recognition, OSR) является фундаментальной проблемой в области биометрической идентификации и компьютерного зрения. В отличие от задачи закрытого распознавания, где система должна классифицировать объекты только из заранее известных классов, OSR требует от системы не только идентифицировать известные объекты, но и надежно отвергать объекты, не принадлежащие ни одному из известных классов~\cite{scheirer2013toward}. Эта задача критически важна для реальных систем безопасности, таких как контроль доступа, паспортный контроль и идентификация животных~\cite{gunther2017toward, patton2023deep}.

Традиционный подход к решению задачи OSR состоит в обучении модели, которая производит дискриминативные векторы признаков (эмбеддинги), и использовании косинусного расстояния между этими эмбеддингами для оценки семантического различия между образцами~\cite{deng2019arcface, cosface}. Пробный образец отвергается, если его расстояние до каждого объекта в галерее известных лиц превышает определенный порог. В противном случае он принимается, и ему присваивается метка самого близкого объекта в галерее~\cite{uncertainty_estimation_in_face_recognition}. Однако, детерминированные эмбеддинги не обладают достаточной устойчивостью к поврежденным пробным образцам. В случае домена изображений, эмбеддинги могут непредсказуемо смещаться при отсутствии некоторых визуальных признаков, что приводит к ошибкам распознавания~\cite{uncertainty_estimation_in_face_recognition}. Для решения этой проблемы были предложены вероятностные эмбеддинги, такие как PFE~\cite{pfe} и SCF~\cite{scf}, которые позволяют вычислять неопределенность, связанную с качеством входного сигнала, и использовать ее для улучшения агрегации шаблонов или фильтрации низкокачественных образцов~\cite{uncertainty_estimation_in_face_recognition, scf}.

Несмотря на успехи, существующие методы OSR, как правило, сосредоточены на снижении ошибок распознавания, игнорируя оценку неопределенности~\cite{uncertainty_estimation_in_face_recognition}. Это происходит потому, что существующие оценки неопределенности не обладают свойствами, необходимыми для надежной системы OSR. Например, оценка неопределенности с помощью вероятностных эмбеддингов для измерения качества образца разумна, но она не покрывает все причины ошибок OSR. Система OSR должна демонстрировать высокую неопределенность, когда эмбеддинг пробного образца близок к нескольким классам в галерее. В таких случаях неопределенность, предсказанная SCF~\cite{scf}, может быть относительно низкой и не отражать неопределенность, связанную с неоднозначностью идентификации~\cite{uncertainty_estimation_in_face_recognition}.

\section{Оценка неопределенности в открытой установке распознавания}
\label{sec:uncertainty_osr}

Оценка неопределенности играет жизненно важную роль в системах распознавания, ориентированных на риск~\cite{abdor2021review}. В OSR неопределенность может возникать из двух основных источников: неопределенность данных (из-за зашумленных или поврежденных входных образцов) и неопределенность модели (из-за несовершенного обучения или ограниченных данных)~\cite{hullermeier2021aleatoric}. Хотя неопределенность модели часто решается с помощью ансамблей или байесовских нейронных сетей~\cite{gal2016dropout, maddox_2019_simple}, наша работа сосредоточена на неопределенности данных — которая более непосредственно управляема во время вывода.

Существующая литература по OSR обычно рассматривает неопределенность как средство для обнаружения образцов вне галереи~\cite{bendale2016towards}, а не как предиктор всех трех типов ошибок OSR: ложного принятия, ложного отказа и неправильной идентификации~\cite{uncertainty_estimation_in_face_recognition}. Как показано на Рис.~\ref{fig:errors}, эти ошибки имеют различные причины — и ни один из существующих методов не охватывает их все. Например, SCF~\cite{scf} отлично справляется с обнаружением низкокачественных образцов, но не способен помечать неоднозначные пробные образцы, находящиеся рядом с границами классов. Напротив, методы, основанные на неопределенности оценки принятия~\cite{huber2022stating}, обнаруживают образцы, пересекающие границы, но игнорируют дисперсию, вызванную искажениями.

Наша работа заполняет этот пробел, предлагая единый фреймворк, который учитывает как качество образца, так и структуру галереи — комбинация, ранее не исследованная в литературе по OSR~\cite{uncertainty_estimation_in_face_recognition}.

\section{Вероятностные эмбеддинги и метрическое обучение}
\label{sec:prob_embeddings}

Для моделирования неопределенности данных в последнее время предлагаются вероятностные эмбеддинговые модели, которые предсказывают распределения в пространстве признаков, а не точечные оценки. PFE~\cite{pfe}, SCF~\cite{scf} и ScaleFace~\cite{kail2023scaleface} являются яркими примерами. Эти модели оценивают дисперсию эмбеддинга (через параметры концентрации или масштаба) и используют ее для улучшения агрегации шаблонов, фильтрации низкокачественных образцов или вычисления расстояний, учитывающих неопределенность~\cite{uncertainty_estimation_in_face_recognition}.

Однако эти методы рассматривают неопределенность исключительно как зависящую от образца — игнорируя геометрическую структуру галереи. Например, SCF предсказывает распределение фон Мизеса-Фишера для каждого образца, но не учитывает, как его среднее соотносится с соседними классами. Это приводит к сбоям, когда высококачественные образцы оказываются рядом с границами решений — распространенная причина неправильной идентификации в реальных развертываниях~\cite{uncertainty_estimation_in_face_recognition}.

Наша рамка HolUE расширяет эти модели, интегрируя неопределенность, связанную с галереей — что позволяет обнаруживать все три типа ошибок OSR через принципиальный байесовский интеграл (Ур.~\ref{eq:holue_integral})~\cite{uncertainty_estimation_in_face_recognition}.

\section{Методы, учитывающие галерею}
\label{sec:gallery_aware}

Несколько недавних работ предлагают модифицировать саму систему распознавания (эмбеддинговую модель или логику принятия решений) для повышения надежности. Например, Vareto et al.~\cite{VARETO2024104862} улучшают разделение классов с помощью Objectosphere и Maximal Entropy losses; Cao et al.~\cite{haaml} предлагают модель HAAML с угловыми маржами, учитывающими сложность, и фильтрацией шумовых меток; Robbins et al.~\cite{10494315} представляют DaliID, архитектуру, адаптивную к искажениям; Deng et al.~\cite{10030586} рассматривают нераспознаваемые лица как отдельный кластер во время обучения. Однако все эти методы изменяют саму систему распознавания.

В отличие от них, наш метод HolUE работает пост-фактум на фиксированном конвейере OSR, оценивая неопределенность без изменения решений распознавания~\cite{uncertainty_estimation_in_face_recognition}.

\section{Пробел в исследованиях и наш вклад}
\label{sec:research_gap}

На наш взгляд, в существующей литературе отсутствует метод, который бы совместно моделировал как неопределенность эмбеддингов, так и неопределенность, вызванную структурой галереи, в рамках единой вероятностной модели. Мы решаем этот пробел, предлагая HolUE — целостный оценщик неопределенности, основанный на байесовской модели, которая вычисляет апостериорное распределение классов $p(c|x)$ путем интегрирования по пространству эмбеддингов (Ур.~\ref{eq:holue_integral})~\cite{uncertainty_estimation_in_face_recognition}.

Объединяя информацию о качестве образца ($p(z|x)$) и положении эмбеддинга в галерее ($p(c|z)$), HolUE предоставляет единый показатель неопределенности, который обнаруживает все три типа ошибок OSR — как продемонстрировано на Рис.~\ref{fig:errors} и подтверждено экспериментами в Разделе~\ref{sec:experiments}.

Кроме того, мы показываем, что наш метод обобщается за пределы распознавания лиц — достигая передового уровня производительности на наборах данных для аудио (VoxBlink~\cite{voxblink}) и идентификации животных (Whale~\cite{cheeseman2022happywhale}). Это позиционирует HolUE как универсальное решение для OSR, контролируемого по риску~\cite{uncertainty_estimation_in_face_recognition}.

% \begin{figure}[htbp]
%     \centering
%     \includegraphics[width=0.8\linewidth]{Dissertation/images/holue_description.png}
%     \caption{Иллюстрация трех типов ошибок в OSR. Синий цвет показывает неопределенность, связанную с галереей (энтропию $p(c|z)$). Зеленый цвет показывает плотность распределения эмбеддингов $p(z|x)$. Символы $\blacklozenge$, $\star$ и $\times$ обозначают места, где HolUE обнаруживает ошибки: неправильную идентификацию, ложное принятие и ложный отказ соответственно.}
%     \label{fig:errors}
% \end{figure}
