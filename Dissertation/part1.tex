\chapter{Обзор существующих подходов к открытому распознаванию и оценке неопределённости}\label{ch:related_work}

\section{Открытая установка распознавания}
\label{sec:osr}
\section{Постановка задачи открытого распознавания}\label{sec:related_work/osr_formulation}
\textcolor{green}{Глава 1 посвящена обзору существующих подходов к решению задачи открытого распознавания (Open-Set Recognition, OSR) и методов оценки неопределённости, применяемых в этой области.}  
\textcolor{blue}{В отличие от классической закрытой постановки распознавания, где все входные образцы предполагаются принадлежащими одному из известных классов, задача OSR предполагает возможность поступления запросов от субъектов, не представленных в галерее — так называемых «вне-галерейных» (out-of-gallery) объектов.}  
\textcolor{green}{Такая постановка характерна для практических биометрических систем, систем видеонаблюдения, аудиоидентификации и других приложений, где надёжное отклонение неизвестных субъектов столь же важно, как и корректная идентификация известных.}  

\textcolor{blue}{Современные методы OSR, как правило, основаны на обучении дискриминативных эмбеддингов — компактных векторных представлений входных данных в метрическом пространстве, где семантически близкие объекты оказываются пространственно близкими.}  
\textcolor{green}{Наиболее известными архитектурами, обеспечивающими высокое качество эмбеддингов, являются ArcFace~\cite{deng2019arcface}, CosFace~\cite{wang2018cosface} и их модификации, такие как HAAML~\cite{cao2025open}.}  
\textcolor{red}{Эти методы оптимизируют угловые расстояния между классами с помощью специализированных функций потерь, что позволяет эффективно разделять известные классы даже в условиях большого масштаба (сотни тысяч классов).}  

\textcolor{blue}{Процедура принятия решения в типичной системе OSR включает два этапа: (1) вычисление максимальной косинусной близости эмбеддинга запроса к центрам классов галереи; (2) сравнение этой близости с порогом принятия/отклонения.}  
\textcolor{green}{Если максимальная близость превышает порог — запрос принимается и класс с максимальной близостью объявляется победителем; иначе запрос отклоняется как вне-галерейный.}  
\textcolor{red}{Однако такой подход не учитывает структуру галереи: например, запрос может иметь высокую близость сразу к нескольким классам, что создаёт неоднозначность идентификации, но не будет отражено в простом косинусном скоре.}  

\textcolor{blue}{Для повышения робастности к повреждённым или низкокачественным входным данным были предложены вероятностные эмбеддинги — модели, в которых каждое входное наблюдение описывается не точечным вектором, а распределением в признаковом пространстве.}  
\textcolor{green}{Среди наиболее заметных работ в этой области — Probabilistic Face Embeddings (PFE)~\cite{shi2019probabilistic}, Spherical Confidence Face (SCF)~\cite{li2021spherical} и ScaleFace~\cite{kail2023scaleface}.}  
\textcolor{blue}{Эти методы моделируют неопределённость, связанную с качеством входного сигнала (data uncertainty), например, из-за размытости, освещения, шума или окклюзий, и позволяют отфильтровывать низкокачественные запросы до этапа распознавания.}  

\textcolor{red}{Однако, как показано в работах~\cite{erlygin2025holue, karpukhin2023probabilistic}, такие методы не способны обнаруживать ошибки, вызванные неоднозначностью галереи: когда запрос качественный, но его эмбеддинг оказывается близок к нескольким классам или на границе принятия/отклонения.}  
\textcolor{blue}{В таких сценариях система может уверенно принять неверное решение, несмотря на наличие высокой неопределённости на уровне семантической структуры задачи.}  

\textcolor{green}{Другой подход к OSR заключается в модификации самой процедуры обучения для лучшего разделения внутри- и вне-галерейных данных.}  
\textcolor{blue}{Например, метод Objectosphere~\cite{vareto2024open} добавляет в функцию потерь регуляризацию, стремящуюся «выталкивать» вне-галерейные образцы за пределы единичной сферы.}  
\textcolor{red}{Метод DaliID~\cite{robbins2024daliid} вводит адаптивные искажения во время обучения, чтобы модель была устойчива к реалистичным артефактам.}  
\textcolor{green}{Подход~\cite{deng2023harnessing} предлагает рассматривать нераспознаваемые лица как отдельный кластер и использовать их для улучшения обобщающей способности модели.}  

\textcolor{blue}{Все перечисленные методы направлены на улучшение точности распознавания, но не предоставляют обобщённой меры неопределённости, пригодной для фильтрации всех трёх типов ошибок OSR: false acceptance, false rejection и misidentification.}  
\textcolor{red}{Более того, большинство из них требуют переобучения базовой модели распознавания, что ограничивает их применимость в сценариях, где основная модель фиксирована (например, в промышленных системах).}  

\textcolor{green}{В отличие от них, пост-хок методы оценки неопределённости, такие как AccScr~\cite{huber2022stating}, основанные на расстоянии до порога принятия, позволяют работать с фиксированным пайплайном.}  
\textcolor{blue}{AccScr интерпретирует разницу между скором принятия и порогом как меру уверенности: чем ближе скор к порогу — тем выше неопределённость.}  
\textcolor{red}{Такой подход эффективно обнаруживает false acceptance и false rejection, но не учитывает относительное положение эмбеддинга запроса относительно других классов, и поэтому не способен детектировать misidentification.}  

\textcolor{green}{Таким образом, в современной литературе отсутствует единый метод, объединяющий информацию о качестве входного сигнала и структуре галереи в единую, теоретически обоснованную меру неопределённости.}  
\textcolor{blue}{Именно этот пробел и призвана закрыть представленная в диссертации работа: предложенный метод HolUE впервые интегрирует обе составляющие неопределённости в рамках единой байесовской модели.}  
\textcolor{red}{Следующие главы посвящены формальному описанию этой модели, её компонентов (GalUE и SCF) и экспериментальной верификации её эффективности.}