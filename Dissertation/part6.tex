% \chapter{Результаты решения обратной задачи выбранным алгоритмом в отдельных случаях}\label{ch:ch_my_6}
% \section{Градиентные методы}\label{sec:ch_my_6/sec1}


\textcolor{green}{Глава 6 посвящена обобщению полученных результатов, обсуждению теоретических и практических аспектов предложенного метода HolUE, а также определению перспектив его дальнейшего развития и применения.}  
\textcolor{blue}{Разработанный в диссертационной работе подход к оценке неопределённости в задаче открытого распознавания демонстрирует, что успешное решение этой проблемы требует учёта двух фундаментальных источников неоднозначности: качества входного сигнала и структуры галереи известных объектов.}  
\textcolor{green}{Предложенные методы GalUE и HolUE впервые формализуют и объединяют эти источники в единой байесовской вероятностной модели, что позволяет обеспечить комплексную и теоретически обоснованную оценку достоверности распознавания.}  

\textcolor{blue}{Одним из ключевых преимуществ HolUE является его пост-хок природа: метод не требует переобучения базовой модели распознавания и может быть интегрирован в уже существующие промышленные пайплайны, основанные на детерминированных эмбеддингах.}  
\textcolor{green}{Это делает его особенно привлекательным для реальных систем, где модификация основной модели может быть затратной или технически невозможной.}  
\textcolor{red}{Более того, эксперименты показывают, что HolUE сохраняет своё превосходство независимо от выбора backbone-архитектуры (ArcFace, CosFace, HAAML), что подтверждает его доменную универсальность.}  

\textcolor{blue}{Важным теоретическим результатом является эквивалентность правила принятия решений в GalUE и классического подхода на основе порога косинусного расстояния.}  
\textcolor{green}{Это позволяет рассматривать GalUE не как альтернативный алгоритм распознавания, а как расширенную интерпретацию уже существующих методов, обогащённую вероятностной семантикой.}  
\textcolor{red}{Такой подход обеспечивает совместимость с устоявшейся практикой, одновременно открывая путь к корректной оценке неопределённости — компоненте, долгое время игнорировавшейся в прикладных системах.}  

\textcolor{green}{Практическое значение работы многократно усиливается её применимостью за пределами биометрии человека.}  
\textcolor{blue}{Успешные результаты на датасетах VoxBlink (распознавание дикторов) и Whale (идентификация китов и дельфинов) демонстрируют, что HolUE является доменно-агностичным фреймворком, пригодным для широкого спектра задач идентификации.}  
\textcolor{red}{Это особенно ценно для приложений в экологии, зоологии и других областях, где сбор больших однородных обучающих наборов затруднён, а надёжность принятия решений критична.}  

\textcolor{green}{Тем не менее, предложенный метод имеет и определённые ограничения.}  
\textcolor{blue}{Вычислительная сложность HolUE линейно зависит от размера галереи, поскольку требуется вычисление правдоподобия для каждого класса при расчёте KL-дивергенции.}  
\textcolor{red}{В системах с миллионами зарегистрированных субъектов это может стать узким местом, требующим разработки приближённых алгоритмов или иерархических стратегий поиска.}  

\textcolor{green}{Другим ограничением является зависимость качества оценки от точности модели вероятностного эмбеддинга (например, SCF).}  
\textcolor{blue}{Как показано в Главе 5, некалиброванные оценки качества, вызванные тренировочным смещением, напрямую снижают эффективность HolUE.}  
\textcolor{red}{Это подчёркивает важность совместной оптимизации пайплайна распознавания и модуля оценки неопределённости, а не их раздельного рассмотрения.}  

\textcolor{green}{В перспективе, метод HolUE может быть расширен в нескольких направлениях.}  
\textcolor{blue}{Во-первых, он естественным образом обобщается на мультимодальные данные: если для каждой модальности (изображение, аудио, текст) доступна своя модель вероятностного эмбеддинга, их можно объединить на уровне интеграла (1) для получения совместной оценки неопределённости.}  
\textcolor{red}{Во-вторых, оценка неопределённости HolUE может быть использована как критерий для активного обучения: система сможет запрашивать метки именно для тех запросов, которые вызывают наибольшую неоднозначность.}  

\textcolor{green}{В-третьих, в контексте больших языковых моделей (LLM), где проблема «галлюцинаций» является одной из центральных, HolUE может быть адаптирован для оценки достоверности генерируемых ответов.}  
\textcolor{blue}{Здесь галереей могут выступать наборы авторитетных источников, а эмбеддингом запроса — скрытое состояние модели.}  
\textcolor{red}{Такая интерпретация открывает путь для создания прозрачных и надёжных ИИ-систем, способных честно сигнализировать о границах своей компетентности.}  

\textcolor{green}{Таким образом, представленная в диссертации работа не только решает конкретную прикладную задачу, но и формирует общий принцип — необходимость комплексного, байесовского взгляда на неопределённость в системах распознавания.}  
\textcolor{blue}{Этот принцип применим далеко за пределами узкой постановки OSR и может стать основой для следующего поколения надёжных и доверенных интеллектуальных систем.}

