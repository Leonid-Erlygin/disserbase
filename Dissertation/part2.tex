\chapter{Математическая модель оценки неопределённости, основанная на структуре галереи (GalUE)}\label{ch:galue}
\section{Байесовская формализация задачи OSR с вне-галерейным классом}\label{sec:galue/bayesian_formulation}
\textcolor{gray}{Глава 2 посвящена разработке математической модели оценки неопределённости, учитывающей структуру галереи известных объектов — так называемой галерею-осознанной неопределённости (gallery-aware uncertainty).}  
\textcolor{blue}{Предлагаемый в этой главе метод GalUE (Gallery-Aware Uncertainty Estimation) представляет собой вероятностную байесовскую модель, которая формализует задачу открытого распознавания как задачу классификации с дополнительным «вне-галерейным» классом, соответствующим неизвестным субъектам.}  
\textcolor{gray}{Такой подход позволяет не только принимать решения о принятии или отклонении запроса, но и оценивать степень неопределённости, связанную с относительным положением эмбеддинга запроса в пространстве галереи.}  

\textcolor{blue}{В рамках модели GalUE предполагается, что входной биометрический образец \(x\) преобразуется в детерминированный эмбеддинг \(z \in \mathbb{S}^{d-1}\) с помощью фиксированной встраиваемой модели (например, ArcFace).}  
\textcolor{gray}{Эмбеддинг \(z\) интерпретируется как наблюдение, на основе которого необходимо оценить апостериорное распределение \(p(c \mid z)\) над множеством возможных классов \(c\), включая как известные классы галереи, так и неизвестные.}  
\textcolor{red}{Для этого сначала задаются априорное распределение \(p(c)\) и класс-условные распределения \(p(z \mid c)\), после чего применяется правило Байеса для вычисления апостериорной вероятности.}  
\section{Моделирование априорного и класс-условных распределений}\label{sec:galue/prior_and_likelihood}
\textcolor{blue}{Априорное распределение \(p(c)\) моделируется как смесь дискретного и непрерывного компонентов:}  
\textcolor{gray}{\[
p(c) = \frac{1 - \beta}{K} \sum_{i=1}^{K} \delta(c - i) + \beta \cdot \mathbb{I}\{c \in (K, K+1]\},
\]}  
\textcolor{blue}{где \(K\) — число известных классов в галерее, \(\delta(\cdot)\) — дельта-функция Дирака, \(\beta \in (0,1)\) — гиперпараметр, задающий априорную вероятность того, что запрос принадлежит неизвестному субъекту, а \(\mathbb{I}\{\cdot\}\) — индикаторная функция.}  
\textcolor{red}{Такое представление позволяет избежать чрезмерной уверенности при отклонении искажённых, но известных образцов, поскольку неизвестные классы моделируются как непрерывное равномерное распределение на дополнительном интервале.}  

\textcolor{gray}{Для моделирования класс-условных распределений \(p(z \mid c)\) для известных классов используется распределение фон Мизеса–Фишера (von Mises–Fisher, vMF) на единичной сфере \(\mathbb{S}^{d-1}\):}  
\textcolor{gray}{\[
p(z \mid c) = C_d(\kappa) \exp\left(\kappa \mu_c^\top z\right), \quad c \in \{1, \dots, K\},
\]}  
\textcolor{blue}{где \(\mu_c\) — центр класса \(c\) (агрегированный эмбеддинг шаблона в галерее), \(\kappa > 0\) — параметр концентрации, общий для всех классов, а \(C_d(\kappa)\) — нормировочная константа.}  
\textcolor{red}{Для вне-галерейных классов предполагается равномерное распределение эмбеддингов по сфере, что эквивалентно заданию \(p(z \mid c) = 1 / S_{d-1}\) для \(c \in (K, K+1]\), где \(S_{d-1}\) — площадь единичной сферы в \(\mathbb{R}^d\).}  

\section{Вывод апостериорного распределения и оценка неопределенности на его основе}\label{sec:galue/posterior_and_decision}
\textcolor{gray}{С использованием заданных \(p(c)\) и \(p(z \mid c)\), апостериорное распределение \(p(c \mid z)\) вычисляется по формуле Байеса:}  
\textcolor{gray}{\[
p(c \mid z) = \frac{p(z \mid c) p(c)}{p(z)},
\]}  
\textcolor{blue}{где маргинальная вероятность \(p(z)\) даётся выражением}  
\textcolor{gray}{\[
p(z) = \frac{1 - \beta}{K} \sum_{i=1}^{K} p(z \mid c=i) + \frac{\beta}{S_{d-1}}.
\]}  
\textcolor{red}{Это выражение следует непосредственно из интегрирования по смешанному пространству классов и подтверждается в Приложении к статье в IEEE Access.}  

\textcolor{gray}{На основе полученного апостериорного распределения принимается решение: запрос отклоняется, если вероятность вне-галерейного класса превышает вероятности всех известных классов; в противном случае присваивается метка класса с максимальной апостериорной вероятностью.}  
\textcolor{blue}{Можно показать, что такое правило принятия решений эквивалентно классическому подходу на основе косинусного расстояния с порогом \(\tau = f(\kappa, \beta, K)\), что обеспечивает совместимость с существующими системами распознавания.}  

\textcolor{gray}{Однако ключевое преимущество GalUE заключается не в изменении решения, а в возможности оценки неопределённости.}  
\textcolor{blue}{В качестве меры неопределённости предлагается использовать максимальную апостериорную вероятность:}  
\textcolor{gray}{\[
q_{\text{GalUE}}(z) = \max_{c \in \{0,1,\dots,K\}} p(c \mid z),
\]}  
\textcolor{red}{где \(c=0\) обозначает объединённый вне-галерейный класс. Чем меньше эта величина, тем выше неопределённость системы.}  

\textcolor{gray}{Такая мера естественным образом отражает два типа ошибок:}  
\textcolor{blue}{— при ложном принятии (false acceptance) апостериорная вероятность вне-галерейного класса близка к вероятности ближайшего известного класса, что снижает \(q_{\text{GalUE}}\);}  
\textcolor{blue}{— при ошибке идентификации (misidentification) эмбеддинг запроса оказывается между двумя или более известными классами, что также приводит к снижению максимальной вероятности.}  
\textcolor{red}{В то же время, GalUE не способен обнаруживать ошибки ложного отказа (false rejection), вызванные низким качеством входного сигнала, поскольку предполагает детерминированный эмбеддинг.}  

\textcolor{gray}{Таким образом, метод GalUE обеспечивает теоретически обоснованную оценку галерею-осознанной неопределённости и формирует основу для последующего объединения с информацией о качестве сигнала в рамках комплексной модели HolUE, описанной в Главе 3.}